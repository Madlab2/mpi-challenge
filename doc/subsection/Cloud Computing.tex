\subsection{Cloud Computing}
Der Begriff Cloud Computing beschreibt ein Modell, über den Zugriff auf verschiedene Rechenressourcen, welche in einem Rechenzentrum gehostet werden. Der Zugriff erfolgt durch die Verwendung von Ressourcen aus einer geteilten Gesamtheit des Service Providers. Cloud Computing zeichnet sich besonders durch die flexible Aufteilung der vorhandenen Ressourcen auf mehrere Nutzer aus. Diese sind meist virtualisiert und geben dem Nutzer keinen oder nur beschränkte Kontrolle und Wissen über den Ort der verwendeten Ressourcen. Dynamisches Freigegeben und Besetzen ermöglichen optimale Ausnutzung der vorhandenen Kapazitäten und ermöglichen Nutzern flexible zusätzliche Ressourcen nach Bedarf zu besetzen. Es gibt drei verschiedene Modelle, nach denen ein Cloud Computing Service aufgebaut sein kann. \textit{Software as a Service} ermöglicht es vom Anbieter bereitgestellte Software auf Cloud Hardware auszuführen. \textit{Platform as a Service} ist ein Modell bei dem den Nutzern verschiedene Instanzen in der Cloud Platform zur Verfügung gestellt werden, auf denen Programme basierend auf vom Provider bereitgestellten Programmiersprachen und Bibliotheken verwenden können. Hierbei besteht keine Kontrolle über die hinterliegende Cloud Struktur. \textit{Platform as a Service} ermöglicht den Nutzern mehr Kontrolle über Ihre Instanzen in Form vom Betriebssystem, Applikationen und Speicher.\cite{b2} Viele Provider bieten eine Mischung dieser drei Arten an und sind je anch Ausrichtung unterschiedlich gewichtet.//
Im Rahmen dieses Projektes wird die BW Cloud verwendet. Diese ist ein Cloud-Service bereitgestellt vom Land Baden-Württemberg für Forschung und Lehre, welcher von mehreren Rechenzentren an Hochschulen in Baden-Württemberg gehostet wird. Hierbei handelt es sich um einen Platform as a Service Cloud Computing Dienstleister. Den Nutzern wird bei der Einrichtung von Instanzen Freiheiten zur Wahl des Betriebssystems und der Netzwerkstruktur gegeben, was sie sehr flexibel für verschiedene Verwendungen einsetzbar macht. Es ist möglich die Instanzen intern zu vernetzen, oder mit dem öffentlichen Netzzugriff zu verbinden. Dabei ist zu beachten, dass nur eine begrenzte Anzahl und Größe von Instanzen jedem Standard Nutzer zur Verfügung steht. In diesem Fall werden Vier Instanzen pro Nutzer erstellt. Daher stehen 12 Instanzen der Gruppe zur Verfügung. Die Instanzen sind über das öffentliche Netzwerk miteinander verbunden, da die direkte Verknüpfung von Instanzen verschiedener Nutzer nur durch manuelle Freischaltung durch die BW Cloud möglich ist. Zugriff der verschiedenen Instanzen aufeinander wird durch SSH realisiert.
\\
  

