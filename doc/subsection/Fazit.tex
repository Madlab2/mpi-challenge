\section{Fazit}
%\label{section:Fazit}
Durch die Gegenüberstellung der Laufzeiten des Merge-Sort-Algorithmus ohne und mit Parallelisierung konnten für die Autoren wertvolle edukative Erkenntnisse erworben werden. Folgende in der Vorlesung vermittelte Vor- und Nachteile von Parallelisierung konnten nachvollzogen werden:
\\
Bei der nicht-optimierten, parallelen Implementierung benötigt das Sortieren länger als ohne Parallelisierung. So ist erst ab fünf Rechenknoten die parallelisierte Variante mit MPI gleich schnell wie die nicht-parallelisierte Variante. Erst durch die diskutierte Optimierung ist eine Laufzeitverbesserung festzustellen, sodass zwei Rechenknoten deutlich schneller als der nicht-parallelisierte Merge Sort sind. In diesem Fall kann anhand der Laufzeitmessung eine Laufzeitverbesserung schon bei niedrigen Rechenknotenzahlen festgestellt werden. Mit steigender Rechenknotenzahl verändert sich die Laufzeit nicht mehr messbar und es findet keine Laufzeitverbesserung mehr statt.\\
Eine weitere Erkenntnis ist, dass die Laufzeit bei gleichbleibenden Rechenknotenzahlen und Wörterzahlen nicht deterministisch ist. Beim mehrfachen Aufrufen des Merge-Sorts mit MPI variiert die Laufzeit bei gleicher Dateigröße zum Teil signifikant. Dieses Phänomen kann durch die Laufzeitumgebung in einer Cloud erklärt werden.\\
Zeitlich konnte der Optimierungsschritt III nicht umgesetzt werden. Die Autoren gehen davon aus, dass die Laufzeit mit Optimierung III weiter verbessert werden kann.
\\
Abschließend kann behauptet werden, dass im Rahmen des Projektes Vorteile als auch Problemzonen von Parallelisierung aufgezeigt und demonstriert wurden. So kann Parallelisierung des Merge-Sort-Algorithmus einen signifikanten Laufzeitvorteil bieten, allerdings auf Kosten der Effizienz des Programmes.