In der Informatik, aber auch anderen Bereichen,  müssen häufig Daten nach bestimmten Attributen sortiert werden. Liegen diese in digitaler Form vor, kann eine Sortierung der Daten über Sortieralgorithmen gelöst werden.  Bei kleinen Datensätzen ist die Wahl des Sortieralgorithmus aufgrund der geringen Laufzeit nahezu irrelevant.  Steigt die Datensatzgröße verlängert sich dementsprechend die Sortierzeit.  Zur Verkürzung dieser können effiziente Sortierverfahren wie Merge-Sort verwendet werden.  In der heutigen Zeit, in der die Menge der Daten, z. B. Messreihen aus Versuchen, im Verhältnis zur möglichen Rechenleistung eines Computers massiv gestiegen ist, ist eine effiziente Sortierung von hoher Bedeutung. Zur weiteren Optimierung der Sortierung kann Cloud-Computing eingesetzt werden. Dabei werden Rechenaufgaben nicht auf einem einzelnen Computer ausgeführt, sondern können unter Einsatz des Message Passing Interface (MPI) auf mehrere sog. Maschinen verteilt werden. In der Arbeit wird untersucht ob eine Aufteilung der Sotieraufgabe von Wörtern unter EInsatz der BW-Cloud zu einer Verringerung der Laufzeit betragen kann.