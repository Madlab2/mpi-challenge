Sowohl in der Informatik als auch in anderen Bereichen müssen häufig Daten nach bestimmten Attributen sortiert werden. Liegen diese in digitaler Form vor, kann eine Sortierung der Daten über Sortieralgorithmen gelöst werden. Bei kleinen Datensätzen ist die Wahl des Sortieralgorithmus aufgrund der geringen Laufzeit nahezu irrelevant. Steigt die Datensatzgröße, so verlängert sich die Sortierzeit. Zur Verkürzung dieser können effiziente Sortierverfahren wie Merge-Sort verwendet werden. In der heutigen Zeit ist die Menge der anfallenden Daten, wie Messreihen aus wissenschaftlichen oder technischen Versuchen im Verhältnis zur verfügbaren Rechenleistung eines Computers massiv gestiegen. Daher ist eine effiziente Möglichkeit für Sortierung von hoher Bedeutung. Zur darüber hinausgehenden Laufzeitoptimierung kann Cloud-Computing eingesetzt werden. Hierbei werden Rechenaufgaben nicht auf einem einzelnen Computer ausgeführt, sondern können unter Einsatz von Parallelisierungswerkzeugen wie das Message Passing Interface (MPI) auf mehrere Maschinen (genannt Knoten) verteilt werden. In dieser Arbeit wird am Beispiel der alphabetischen Wörtersortierung untersucht, ob eine Aufteilung der Sortieraufgabe zu einer Senkung der Laufzeit beitragen kann. Hierzu wird der Sortieralgorithmus parallelisiert und in einer Cloud-Umgebung verteilt ausgeführt.