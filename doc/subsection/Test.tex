\subsection{Test}
Für das Projekt werden parallel zur Entwicklung der Funktionen Unit Tests mitverfasst. Hierfür verwendet wird die Testsuite \textit{gtest}, welche in \textit{GoogleTest} enthalten ist. Der Zweck der Unit Tests beschränkt sich auf eine Absicherung der Lauffähigkeit und Gewährleistung der Grundfunktionalität des Programms.
\\
Einige Funktionen, welche komplexe Algorithmen enthalten, werden mithilfe von im Vorhinein definierten \textit{Blackbox-Tests} entwickelt. Beispiele sind die Funktionen \textit{split\_even()} und \textit{merge\_back()}. Es werden jeweils ein Standard-Fall und mehrere Randfälle aus verschiedenen Äquivalenzklassen in die Tests aufgenommen. 
Wenn die zu entwickelnden Funktionen diese Tests bestehen, wird angenommen, dass die geforderte Grundfunktionalität erfüllt wird. Robustheit ist nicht gegeben. Eine vollständige Testabdeckung,
wie z.B. die \textit{vollständige Anweisungsüberdeckung} (C0-Überdeckung), wird nicht realisiert.
\\
Auf eine vollständige Testabdeckung nach einer der bekannten Klassen C0...C3 für strukturelle Tests wird generell verzichtet. Gründe hierfür sind die beschränkte Bearbeitungszeit für das Projekt sowie dass eine vollständige Abdeckung nicht zielführend für das Projekt ist. Das Testen von Nutzereingaben im Programm oder der MPI-Funktionalität würde einen hohen Aufwand beim Testentwurf bedeuten. Um die MPI-Funktionalitäten rigoros zu testen, müssten u.a. der Einfluss der Anzahl verwendeter Rechenknoten, die Behandlung von Knoten-Ausfällen sowie die Auswirkungen maximaler Nachrichtenpaketgrößen betrachtet werden. Hiervon wird abgesehen, da die im Rahmen des Projekts entworfene Software nicht produktiv eingesetzt wird und ausschließlich der Übung mit sowie der Demonstration von Vor- und Nachteilen von MPI dient. Die Vorteile einer vollständigen Testabdeckung würden in keinem Verhältnis zum benötigten Aufwand der Testerstellung stehen.
\\
Wäre eine Absicherung der Software durch Tests gewünscht, müssten die eigens entwickelten und implementierten Algorithmen rigoroser getestet werden. Für die Funktionen \textit{split\_even()} und \textit{merge\_back()} wird eine \textit{Zweigüberdeckung} (C1) als sinnvoll betrachtet, da diese Funktionen eine zentrale Rolle in der Software einnehmen. Die Funktionalitäten des Merge Sort Algorithmus können weiterhin mit geringer Testabdeckung in Form von Blackbox Tests oder \textit{Anweisungsüberdeckung} (C0) für eine Rekursionsebene des Algorithmus gewährleistet werden.
Dies wird ist durch die allgemein bekannte Korrekheit des Algorithmus gerechtfertigt.