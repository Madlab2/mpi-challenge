\subsection{Optimierung}
%\label{subsection:Optimierung}
Im Projekt wurden drei Optimierungen vorgenommen, um die Laufzeit zu verkürzen.\\

1. Optimierungsschritt: Um die Laufzeit des Programms zu verkürzen, werden nicht mehr die einzelnen Chars, die ein Wort bilden verschickt, sondern ein Array  mit allen Chars bzw. Wörtern zusammengesetzt. Um zu verdeutlichen aus welchen Chars ein Wort besteht, wird zwischen den Wörter ein Semikolon eingefügt. Dadurch kann der Slave das Array nach den Wörtern aufteilen. Dies führt zu einer Optimierung der Laufzeit, da die Befehle von MPI\_Send() und MPI\_Recv() nicht mehr nach jedem Char ausgeführt werden sondern nur noch einmal pro Prozess. \\
Wenn die Chars aus der einzulesenden Datei einzeln verschickt werden, sind es ca. 2.465.500 Aufrufe der Funktionen MPI\_Send() und MPI\_Recv(). Zusätzlich werden auch beim Rücksenden der einzelnen Zeichen an den Master ca. 2.465.500 Aufrufe der beiden Funktionen benötigt.\\
Nach der Optimierung waren Verbesserungen der Laufzeit zu beobachten. Zwei Prozesse benötigten nun insgesamt 2,5 Sekunden zum Sortieren. Vor der Optimierung lag die Laufzeit bei vier Prozessen zum Vergleich bei etwa 3 Sekunden.  

2. Optimierungsschritt: Im ersten Programmentwurf versendet der Master die Wörter an die fünfzehn Prozesse zum Sortieren, ohne selber einen Bereich zu sortieren. Dies führt dazu, dass ein Prozess nur die Daten verschickt und empfängt. In der Zwischenzeit, wo die fünfzehn Prozesse sortieren, wartet ein Prozess bis Daten zurückgeschickt werden. Diese Zwischenzeit kann der Master nutzen, um selber einen Teilbereich zu sortieren. So werden sechzehn statt fünfzehn Prozesse zum Sortieren genutzt.  

3. Optimierungsschritt: Eine weiterer Optimierungsschritt ist, dass alle eingelesenen Wörter aus der Datei in einem Array verschickt werden. Somit muss der Array zunächst nicht in die einzelnen zu sortierenden Bereiche aufgeteilt werden. Jeder Prozess erhält einen Buchstabenbereich von 2-3 Buchstaben. Z.b. erhält der erste Prozess den Buchstabenbereich von a und b. Somit sortiert er nur die Wörter alphabetisch, die mit a oder b anfangen. Die Optimierung liegt am Zusammenführen der einzeln sortierten Arrays der Slaves. Der Master fügt die sortierten Arrays hintereinander zu einem vollständig Array zusammen, in dem alle Wörter enthalten sind. \\