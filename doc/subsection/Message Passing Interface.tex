\subsection{Message Passing Interface}
%\label{subsection:Message Passing Interface}
Message Passing Interface (MPI) ist ein Standard zum Austausch von Nachrichten zwischen Prozessen auf einem Rechner sowie zwischen Prozessen auf verteilten Rechnern. MPI dient als Programmierschnittstelle und verschickt verpackte Nachrichten an parallel laufende Prozesse.\\
Die Bibliotheken OpenMPI und MPICH setzen den MPI Standard für die Programmiersprache C++ um.
Auf den für das Projekt verwendeten Linux Rechnern steht OpenMPI zur Verfügung und wird für die Paralellisierungsanwendungen des Projektes verwendet.\\ 
\subsubsection{Konzept}
Die Grundidee von MPI besteht aus zwei Konzepten: dem Konzept der Prozessgruppe und dem des Kommunikationskontexts. Diese beiden Begriffe werden im Folgenden erläutert.
\\
Mit Prozessgruppen wird die Anzahl an beteiligten Rechnern definiert. Beim Ausführen des Programms mit MPI werden zu Beginn die beteiligten Prozesse gestartet. Innerhalb des Programms können die einzelnen Prozesse gesteuert und zusätzlich in einzelne Untergruppen zusammengefasst werden.
\\
Einzelne Nachrichten können anhand der Empfänger- bzw. Sender-ID zugeordnet werden. Mit MPI können so Nachrichten zu einem bestimmten Zeitpunkt an einen bestimmten Prozess verschickt werden.\\
Der Kommunikationskontext dient als Lösung bei Überschneidungen der Tag-,Sender- und Empfänger-IDs. Jeder Sende- und Empfangsvorgang gehört zu einem Kontext. Die Kommunikation erfolgt ausschließlich über diesen Kontext, sodass es zu keinen Verwechslungen kommen kann.
Durch einen Kommunikator werden Prozessgruppen und Kommunikationskontexte miteinander vereint. Beim Programmstart wird der allgemeine Kommunikator \textit{MPI\_COMM\_WORLD} erzeugt.~\cite{b1}. Dieser wird auch in diesem Projekt verwendet.\\
\subsubsection{Verwendung} Um MPI zu verwenden muss zunächst MPI initialisiert werden, wobei der oben genannte Kommunikator \textit{MPI\_COMM\_WORLD} erzeugt wird.\\
Mit dem Befehl \textit{MPI\_Comm\_size()} kann die Größe der Gruppe ermittelt werden. Mit dem Befehl \textit{MPI\_Comm\_rank()} kann der Rang der einzelnen Prozesse ermittelt werden. Der Rang beschreibt die Positionen der einzelnen Prozesse innerhalb einer Gruppe.
Die MPI-Laufzeitumgebung kann mit \textit{MPI\_Finalize()} am Programmende gestoppt werden.\\
Zur Punkt-zu-Punkt-Kommunikation über MPI werden die Funktionen \textit{MPI\_Send()} und \textit{MPI\_Recv()} verwendet. Dabei werden einzelne Nachrichten von einem Prozess zu einem anderen verschickt. Diese Art des Sendens ist blockierend. \textit{MPI\_Send()} und \textit{MPI\_Recv()} blockieren das System bis eine Nachricht vollständig verschickt bzw. empfangen wird. Sobald die Nachricht verschickt bzw. empfangen wurde, setzt sich der Programmablauf fort. \\
Ein MPI-basiertes Programm kann mit Eingabe des Befehls \textit{mpirun} sowie einem Verweis auf die zu nutzenden Prozesse bzw. Rechner in einer Konsole gestartet werden. Alle Prozesse, auch wenn diese auf physikalisch anderen Rechnern laufen sollen, werden gleichzeitig gestartet~\cite{b1}.