\subsection{Test}
%\label{subsection:Test}
Für das Projekt werden von Beginn an Unit Tests mitgeschrieben. Hierfür verwendet wird die Testsuite gtest, welche in GoogleTest enthalten ist. Der Zweck der Unit Tests beschränkt sich auf eine Absicherung der Lauffähigkeit und Gewährleistung der Grundfunktionalität des Programms; eine vollständige Testabdeckung nach einer der bekannten Klassen C0...C3 für strukturelle Tests wird aus mehreren Gründen verzichtet. Erstens ist die Bearbeitungszeit für das Projekt zu gering, zweitens wäre bei der konkreten Zielsetzung des Projekts eine vollständige Abdeckung nicht unbedingt zielführend. Insbesondere das Testen von Nutzereingaben im Programm sowie der verwendeten MPI-Funktionalität würde einen hohen Aufwand beim Testentwurf bedeuten. Die im Rahmen des Projekts entworfene Software soll allein der Übung mit MPI sowie der Demonstration der Vor- und Nachteile hiervon dienen und nicht in einer kritischen Umgebung ausgeführt werden. Zuletzt ist der Projektumfang gering. Die Vorteile einer vollständige Testabdeckung würden im konkreten Fall in keinem Verhältnis zum Aufwand stehen.
\paragraph
Stattdessen werden Tests für das Projekt allein für "Komfortzwecke" geschrieben. Einige Funktionen, welche nicht mehr auf Anhieb zu verstehende Algorithmen
enthalten, werden mithilfe von im Vorhinein spezifizierten \textit{Blackbox-Tests} entwickelt. Beispiele für solche Funktionen sind \textit{split_even()} und \textit{merge_back()}. Hierfür wurden jeweils ein Standard-Fall und mehrere denkbare Randfälle aus verschiedenen Äquivalenzklassen in Tests formuliert. 
Wenn die zu entwickelten Funktionen diese Tests bestehen, kann davon ausgegangen werden, dass die jeweilige Grundfunktionalität gegeben ist. Robustheit ist hiermit nicht gegeben. Eine vollständige Testabdeckung,
sei es auch "nur" die \textit{vollständige Anweisungsüberdeckung} (C0-Überdeckung), wird nicht realisiert.
\paragraph
Angenommen die Zielsetzung des Projekts sei, die Software produktiv einzusetzen. In diesem Fall würden Tests eine andere Priorität erhalten. Aus der Sicht der Autoren wäre es sinnvoll, die Funktionalitäten des Merge Sort Algorithmus auch weiterhin nur mit geringer Testabdeckung in Form von Blackbox Tests oder \textit{Anweisungsüberdeckung} (C0) für eine Rekursionsebene des Algorithmus zu gewährleisten. 
Dies wird begründet mit der Tatsache, dass der Algorithmus als solcher bekannt ist und seine Funktionstüchtigkeit mehrfach erwiesen wurde. Anders sieht es bei denen von den Autoren eigens entwickelten und implementierten 
Algorithmen und Abläufen aus. Für die oben genannten Funktionen \textit{split_even()} und \textit{merge_back()} wird dringend eine \textit{Zweigüberdeckung} (C1) empfohlen. Dies wird als sinnvoll erachtet, da diese Funktionen eine zentrale Rolle für die Korrektheit und Funktionstüchtigkeit der Software inne haben. Des Weiteren sind die Funktionen von (noch) geringer Komplexität, sodass der zusätzliche Aufwand sich in Grenzen halten sollte. Von einer höheren Testabdeckung, insbesondere bereits von der \textit{C2-Überdeckung} raten die Autoren aufgrund des rasch ansteigenden Aufwands jedoch ab.
\paragraph
Ein besonderer Fokus und gleichzeitig eine besondere Herausforderung stellt die MPI-Funktionalität des Projektes dar. Diese erhöht den Aufwand des Testdesigns um eine ganze Dimension. Es stellen sich (unter anderem) Fragen, ob die Anzahl der verwendeten Rechenknoten einen Einfluss auf den Ablauf des Programms hat, wie Ausfälle von Knoten behandelt werden, ob das Erreichen von möglichen maximalen Größen der Nachrichtenpakete eine Auswirkung auf das Programm hat und berücksichtigt wird. Weiter stellt sich die Frage, ob jegliche Tests, welche MPI-Funktionalität abdecken, mit verschiedenen Anzahlen von Knoten durchgeführt werden sollten, und wenn ja, mit welchen Anzahlen genau. Ein möglicher Ansatz ist hier, im Vorhinein das Intervall der verwendeten Knotenzahl zu bestimmen. Dann können die Tests mehrmals mit einer randomisierten Anzahl an Knoten aus dem Intervall ausgeführt werden, wobei es Vorgaben geben sollte, welche gewährleisten, dass die Randbereiche des Intervalls garantiert abgedeckt werden.